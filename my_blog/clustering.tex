% Options for packages loaded elsewhere
% Options for packages loaded elsewhere
\PassOptionsToPackage{unicode}{hyperref}
\PassOptionsToPackage{hyphens}{url}
\PassOptionsToPackage{dvipsnames,svgnames,x11names}{xcolor}
%
\documentclass[
  10pt,
  letterpaper,
  DIV=11,
  numbers=noendperiod,
  oneside]{scrreprt}
\usepackage{xcolor}
\usepackage[top=20mm,bottom=20mm,left=20mm,right=40mm,heightrounded]{geometry}
\usepackage{amsmath,amssymb}
\setcounter{secnumdepth}{-\maxdimen} % remove section numbering
\usepackage{iftex}
\ifPDFTeX
  \usepackage[T1]{fontenc}
  \usepackage[utf8]{inputenc}
  \usepackage{textcomp} % provide euro and other symbols
\else % if luatex or xetex
  \usepackage{unicode-math} % this also loads fontspec
  \defaultfontfeatures{Scale=MatchLowercase}
  \defaultfontfeatures[\rmfamily]{Ligatures=TeX,Scale=1}
\fi
\usepackage{lmodern}
\ifPDFTeX\else
  % xetex/luatex font selection
\fi
% Use upquote if available, for straight quotes in verbatim environments
\IfFileExists{upquote.sty}{\usepackage{upquote}}{}
\IfFileExists{microtype.sty}{% use microtype if available
  \usepackage[]{microtype}
  \UseMicrotypeSet[protrusion]{basicmath} % disable protrusion for tt fonts
}{}
\makeatletter
\@ifundefined{KOMAClassName}{% if non-KOMA class
  \IfFileExists{parskip.sty}{%
    \usepackage{parskip}
  }{% else
    \setlength{\parindent}{0pt}
    \setlength{\parskip}{6pt plus 2pt minus 1pt}}
}{% if KOMA class
  \KOMAoptions{parskip=half}}
\makeatother
% Make \paragraph and \subparagraph free-standing
\makeatletter
\ifx\paragraph\undefined\else
  \let\oldparagraph\paragraph
  \renewcommand{\paragraph}{
    \@ifstar
      \xxxParagraphStar
      \xxxParagraphNoStar
  }
  \newcommand{\xxxParagraphStar}[1]{\oldparagraph*{#1}\mbox{}}
  \newcommand{\xxxParagraphNoStar}[1]{\oldparagraph{#1}\mbox{}}
\fi
\ifx\subparagraph\undefined\else
  \let\oldsubparagraph\subparagraph
  \renewcommand{\subparagraph}{
    \@ifstar
      \xxxSubParagraphStar
      \xxxSubParagraphNoStar
  }
  \newcommand{\xxxSubParagraphStar}[1]{\oldsubparagraph*{#1}\mbox{}}
  \newcommand{\xxxSubParagraphNoStar}[1]{\oldsubparagraph{#1}\mbox{}}
\fi
\makeatother

\usepackage{color}
\usepackage{fancyvrb}
\newcommand{\VerbBar}{|}
\newcommand{\VERB}{\Verb[commandchars=\\\{\}]}
\DefineVerbatimEnvironment{Highlighting}{Verbatim}{commandchars=\\\{\}}
% Add ',fontsize=\small' for more characters per line
\usepackage{framed}
\definecolor{shadecolor}{RGB}{241,243,245}
\newenvironment{Shaded}{\begin{snugshade}}{\end{snugshade}}
\newcommand{\AlertTok}[1]{\textcolor[rgb]{0.68,0.00,0.00}{#1}}
\newcommand{\AnnotationTok}[1]{\textcolor[rgb]{0.37,0.37,0.37}{#1}}
\newcommand{\AttributeTok}[1]{\textcolor[rgb]{0.40,0.45,0.13}{#1}}
\newcommand{\BaseNTok}[1]{\textcolor[rgb]{0.68,0.00,0.00}{#1}}
\newcommand{\BuiltInTok}[1]{\textcolor[rgb]{0.00,0.23,0.31}{#1}}
\newcommand{\CharTok}[1]{\textcolor[rgb]{0.13,0.47,0.30}{#1}}
\newcommand{\CommentTok}[1]{\textcolor[rgb]{0.37,0.37,0.37}{#1}}
\newcommand{\CommentVarTok}[1]{\textcolor[rgb]{0.37,0.37,0.37}{\textit{#1}}}
\newcommand{\ConstantTok}[1]{\textcolor[rgb]{0.56,0.35,0.01}{#1}}
\newcommand{\ControlFlowTok}[1]{\textcolor[rgb]{0.00,0.23,0.31}{\textbf{#1}}}
\newcommand{\DataTypeTok}[1]{\textcolor[rgb]{0.68,0.00,0.00}{#1}}
\newcommand{\DecValTok}[1]{\textcolor[rgb]{0.68,0.00,0.00}{#1}}
\newcommand{\DocumentationTok}[1]{\textcolor[rgb]{0.37,0.37,0.37}{\textit{#1}}}
\newcommand{\ErrorTok}[1]{\textcolor[rgb]{0.68,0.00,0.00}{#1}}
\newcommand{\ExtensionTok}[1]{\textcolor[rgb]{0.00,0.23,0.31}{#1}}
\newcommand{\FloatTok}[1]{\textcolor[rgb]{0.68,0.00,0.00}{#1}}
\newcommand{\FunctionTok}[1]{\textcolor[rgb]{0.28,0.35,0.67}{#1}}
\newcommand{\ImportTok}[1]{\textcolor[rgb]{0.00,0.46,0.62}{#1}}
\newcommand{\InformationTok}[1]{\textcolor[rgb]{0.37,0.37,0.37}{#1}}
\newcommand{\KeywordTok}[1]{\textcolor[rgb]{0.00,0.23,0.31}{\textbf{#1}}}
\newcommand{\NormalTok}[1]{\textcolor[rgb]{0.00,0.23,0.31}{#1}}
\newcommand{\OperatorTok}[1]{\textcolor[rgb]{0.37,0.37,0.37}{#1}}
\newcommand{\OtherTok}[1]{\textcolor[rgb]{0.00,0.23,0.31}{#1}}
\newcommand{\PreprocessorTok}[1]{\textcolor[rgb]{0.68,0.00,0.00}{#1}}
\newcommand{\RegionMarkerTok}[1]{\textcolor[rgb]{0.00,0.23,0.31}{#1}}
\newcommand{\SpecialCharTok}[1]{\textcolor[rgb]{0.37,0.37,0.37}{#1}}
\newcommand{\SpecialStringTok}[1]{\textcolor[rgb]{0.13,0.47,0.30}{#1}}
\newcommand{\StringTok}[1]{\textcolor[rgb]{0.13,0.47,0.30}{#1}}
\newcommand{\VariableTok}[1]{\textcolor[rgb]{0.07,0.07,0.07}{#1}}
\newcommand{\VerbatimStringTok}[1]{\textcolor[rgb]{0.13,0.47,0.30}{#1}}
\newcommand{\WarningTok}[1]{\textcolor[rgb]{0.37,0.37,0.37}{\textit{#1}}}

\usepackage{longtable,booktabs,array}
\usepackage{calc} % for calculating minipage widths
% Correct order of tables after \paragraph or \subparagraph
\usepackage{etoolbox}
\makeatletter
\patchcmd\longtable{\par}{\if@noskipsec\mbox{}\fi\par}{}{}
\makeatother
% Allow footnotes in longtable head/foot
\IfFileExists{footnotehyper.sty}{\usepackage{footnotehyper}}{\usepackage{footnote}}
\makesavenoteenv{longtable}
\usepackage{graphicx}
\makeatletter
\newsavebox\pandoc@box
\newcommand*\pandocbounded[1]{% scales image to fit in text height/width
  \sbox\pandoc@box{#1}%
  \Gscale@div\@tempa{\textheight}{\dimexpr\ht\pandoc@box+\dp\pandoc@box\relax}%
  \Gscale@div\@tempb{\linewidth}{\wd\pandoc@box}%
  \ifdim\@tempb\p@<\@tempa\p@\let\@tempa\@tempb\fi% select the smaller of both
  \ifdim\@tempa\p@<\p@\scalebox{\@tempa}{\usebox\pandoc@box}%
  \else\usebox{\pandoc@box}%
  \fi%
}
% Set default figure placement to htbp
\def\fps@figure{htbp}
\makeatother


% definitions for citeproc citations
\NewDocumentCommand\citeproctext{}{}
\NewDocumentCommand\citeproc{mm}{%
  \begingroup\def\citeproctext{#2}\cite{#1}\endgroup}
\makeatletter
 % allow citations to break across lines
 \let\@cite@ofmt\@firstofone
 % avoid brackets around text for \cite:
 \def\@biblabel#1{}
 \def\@cite#1#2{{#1\if@tempswa , #2\fi}}
\makeatother
\newlength{\cslhangindent}
\setlength{\cslhangindent}{1.5em}
\newlength{\csllabelwidth}
\setlength{\csllabelwidth}{3em}
\newenvironment{CSLReferences}[2] % #1 hanging-indent, #2 entry-spacing
 {\begin{list}{}{%
  \setlength{\itemindent}{0pt}
  \setlength{\leftmargin}{0pt}
  \setlength{\parsep}{0pt}
  % turn on hanging indent if param 1 is 1
  \ifodd #1
   \setlength{\leftmargin}{\cslhangindent}
   \setlength{\itemindent}{-1\cslhangindent}
  \fi
  % set entry spacing
  \setlength{\itemsep}{#2\baselineskip}}}
 {\end{list}}
\usepackage{calc}
\newcommand{\CSLBlock}[1]{\hfill\break\parbox[t]{\linewidth}{\strut\ignorespaces#1\strut}}
\newcommand{\CSLLeftMargin}[1]{\parbox[t]{\csllabelwidth}{\strut#1\strut}}
\newcommand{\CSLRightInline}[1]{\parbox[t]{\linewidth - \csllabelwidth}{\strut#1\strut}}
\newcommand{\CSLIndent}[1]{\hspace{\cslhangindent}#1}



\setlength{\emergencystretch}{3em} % prevent overfull lines

\providecommand{\tightlist}{%
  \setlength{\itemsep}{0pt}\setlength{\parskip}{0pt}}



 


\KOMAoption{captions}{tablesignature}
\makeatletter
\@ifpackageloaded{caption}{}{\usepackage{caption}}
\AtBeginDocument{%
\ifdefined\contentsname
  \renewcommand*\contentsname{Table of contents}
\else
  \newcommand\contentsname{Table of contents}
\fi
\ifdefined\listfigurename
  \renewcommand*\listfigurename{List of Figures}
\else
  \newcommand\listfigurename{List of Figures}
\fi
\ifdefined\listtablename
  \renewcommand*\listtablename{List of Tables}
\else
  \newcommand\listtablename{List of Tables}
\fi
\ifdefined\figurename
  \renewcommand*\figurename{Figure}
\else
  \newcommand\figurename{Figure}
\fi
\ifdefined\tablename
  \renewcommand*\tablename{Table}
\else
  \newcommand\tablename{Table}
\fi
}
\@ifpackageloaded{float}{}{\usepackage{float}}
\floatstyle{ruled}
\@ifundefined{c@chapter}{\newfloat{codelisting}{h}{lop}}{\newfloat{codelisting}{h}{lop}[chapter]}
\floatname{codelisting}{Listing}
\newcommand*\listoflistings{\listof{codelisting}{List of Listings}}
\makeatother
\makeatletter
\makeatother
\makeatletter
\@ifpackageloaded{caption}{}{\usepackage{caption}}
\@ifpackageloaded{subcaption}{}{\usepackage{subcaption}}
\makeatother
\makeatletter
\@ifpackageloaded{sidenotes}{}{\usepackage{sidenotes}}
\@ifpackageloaded{marginnote}{}{\usepackage{marginnote}}
\makeatother
\usepackage{bookmark}
\IfFileExists{xurl.sty}{\usepackage{xurl}}{} % add URL line breaks if available
\urlstyle{same}
\hypersetup{
  pdftitle={Cluster Analysis with Python},
  pdfauthor={Jesus L. Monroy},
  colorlinks=true,
  linkcolor={blue},
  filecolor={Maroon},
  citecolor={Blue},
  urlcolor={Blue},
  pdfcreator={LaTeX via pandoc}}


\title{Cluster Analysis with Python}
\usepackage{etoolbox}
\makeatletter
\providecommand{\subtitle}[1]{% add subtitle to \maketitle
  \apptocmd{\@title}{\par {\large #1 \par}}{}{}
}
\makeatother
\subtitle{A Basic Example using Polars}
\author{\emph{Jesus L. Monroy}}
\date{Apr, 2025}
\begin{document}
\maketitle

\renewcommand*\contentsname{Table of Contents}
{
\hypersetup{linkcolor=}
\setcounter{tocdepth}{2}
\tableofcontents
}
\listoffigures
\listoftables

\chapter{Cluster Analysis and
Clustering}\label{cluster-analysis-and-clustering}

\textbf{Cluster analysis} encompasses the tools, algorithms, and methods
used to uncover hidden groupings within a dataset based on similarity.

This process includes not only \textbf{clustering} ---\emph{the
application of algorithms like k-means, DBSCAN, or hierarchical
clustering to group data}--- but also the subsequent analysis of the
identified groups' characteristics and properties, ultimately informing
decision-making in domains such as marketing (\emph{customer and product
segmentation}) and finance (\emph{fraud detection}).

\emph{While clustering is a key step in cluster analysis, the terms are
not synonymous}. cluster analysis is the broader framework that includes
interpretation and action based on the discovered groups.

This example is inspired in Machine Learning Mastery blog (Carrascosa
2024)\marginpar{\begin{footnotesize}
\begin{CSLReferences}{2}{0}
\bibitem[\citeproctext]{ref-carrascosa2024}
Carrascosa, Ivan Palomares. 2024. {``Performing Cluster Analysis in
Python: A Step-by-Step Tutorial.''} Posted on Statology.
\url{https://www.statology.org/performing-cluster-analysis-in-python-a-step-by-step-tutorial}.
\end{CSLReferences}
\vspace{2mm}\par\end{footnotesize}}.

\section{Environment settings}\label{environment-settings}

We import necessary libraries.

\begin{Shaded}
\begin{Highlighting}[]
\CommentTok{\# import libraries}
\ImportTok{import}\NormalTok{ numpy }\ImportTok{as}\NormalTok{ np}
\ImportTok{import}\NormalTok{ polars }\ImportTok{as}\NormalTok{ pl}
\ImportTok{import}\NormalTok{ matplotlib.pyplot }\ImportTok{as}\NormalTok{ plt}
\NormalTok{plt.style.use(}\StringTok{\textquotesingle{}ggplot\textquotesingle{}}\NormalTok{)}
\ImportTok{import}\NormalTok{ seaborn }\ImportTok{as}\NormalTok{ sns}
\ImportTok{from}\NormalTok{ sklearn.cluster }\ImportTok{import}\NormalTok{ KMeans}
\ImportTok{from}\NormalTok{ sklearn.preprocessing }\ImportTok{import}\NormalTok{ StandardScaler}
\end{Highlighting}
\end{Shaded}

We specify the url link where the dataset is hosted.

\begin{Shaded}
\begin{Highlighting}[]
\NormalTok{url }\OperatorTok{=} \StringTok{\textquotesingle{}https://tinyurl.com/5n8k4bku\textquotesingle{}}
\end{Highlighting}
\end{Shaded}

We save the dataset in a polars dataframe.

\begin{Shaded}
\begin{Highlighting}[]
\NormalTok{df }\OperatorTok{=}\NormalTok{ pl.read\_csv(url)}
\end{Highlighting}
\end{Shaded}

\newpage{}

\section{Load data}\label{load-data}

In the following table, we can see an extract of the dataset used for
this project.

\begin{Shaded}
\begin{Highlighting}[]
\CommentTok{\# Show dataframe}
\NormalTok{df.head()}
\end{Highlighting}
\end{Shaded}

{
\makeatletter
\def\LT@makecaption#1#2#3{%
  \noalign{\smash{\hbox{\kern\textwidth\rlap{\kern\marginparsep
  \parbox[t]{\marginparwidth}{%
    \footnotesize{%
      \vspace{(1.1\baselineskip)}
    #1{#2: }\ignorespaces #3}}}}}}%
    }
\makeatother

\begin{longtable}[]{@{}lllll@{}}
\caption{Clustering dataset}\tabularnewline
\toprule\noalign{}
CustomerID & Gender & Age & Annual Income (k\$) & Spending Score
(1-100) \\
i64 & str & i64 & i64 & i64 \\
\midrule\noalign{}
\endfirsthead
\toprule\noalign{}
CustomerID & Gender & Age & Annual Income (k\$) & Spending Score
(1-100) \\
i64 & str & i64 & i64 & i64 \\
\midrule\noalign{}
\endhead
\bottomrule\noalign{}
\endlastfoot
1 & "Male" & 19 & 15 & 39 \\
2 & "Male" & 21 & 15 & 81 \\
3 & "Female" & 20 & 16 & 6 \\
4 & "Female" & 23 & 16 & 77 \\
5 & "Female" & 31 & 17 & 40 \\
\end{longtable}

}

\section{Data preparation}\label{data-preparation}

The dataset needs no furhter cleaning, so we just have to do the next
steps:

\begin{itemize}
\tightlist
\item
  Feature selection: select relevant numerical attributes for
  clustering.
\item
  Normalization: scale the values of attributes. This will be helpful
  for the effectiveness of the clustering algorithm.\\
\end{itemize}

\begin{Shaded}
\begin{Highlighting}[]
\CommentTok{\# Select relevant features for clustering}
\NormalTok{X}\OperatorTok{=}\NormalTok{ df.select(}\StringTok{\textquotesingle{}Annual Income (k$)\textquotesingle{}}\NormalTok{, }\StringTok{\textquotesingle{}Spending Score (1{-}100)\textquotesingle{}}\NormalTok{).to\_numpy()}

\CommentTok{\# Feature scaling}
\NormalTok{scaler }\OperatorTok{=}\NormalTok{ StandardScaler()}
\NormalTok{X\_scaled }\OperatorTok{=}\NormalTok{ scaler.fit\_transform(X)}
\end{Highlighting}
\end{Shaded}

\section{Model selection}\label{model-selection}

Many clustering algorithms require setting configuration parameters,
often termed \emph{hyperparameters} in machine learning.

Specifically, the \emph{k-means algorithm}, requires that we
predetermine the number of \emph{clusters} (K) to be identified.

In some cases, domain expertise or existing knowledge about the problem
and data can provide an approximate value for K. However, when such
guidance is lacking, the \textbf{Elbow Method} offers a more systematic
alternative to trial and error.

\emph{This method involves repeatedly applying the k-means algorithm
with a range of increasing K values and then plotting the inertia for
each result}. Inertia serves as a metric for cluster quality, with lower
values signifying more well-defined clusters. The goal is to find a
clustering solution with both low inertia and a reasonable number of
clusters.

Consequently, when the inertia values for various K are visualized as a
curve, the point on the curve closest to the origin (0,0), representing
the trade-off between low inertia and low K, often suggests a good
choice for the number of clusters.

\subsection{Determing the K using the Elbow
Method}\label{determing-the-k-using-the-elbow-method}

\begin{Shaded}
\begin{Highlighting}[]
\CommentTok{\# Determine optimal number of clusters (K)}
\NormalTok{inertia }\OperatorTok{=}\NormalTok{ []}

\ControlFlowTok{for}\NormalTok{ k }\KeywordTok{in} \BuiltInTok{range}\NormalTok{(}\DecValTok{1}\NormalTok{, }\DecValTok{11}\NormalTok{):}
\NormalTok{    kmeans }\OperatorTok{=}\NormalTok{ KMeans(n\_clusters}\OperatorTok{=}\NormalTok{k, random\_state}\OperatorTok{=}\DecValTok{626}\NormalTok{)}
\NormalTok{    kmeans.fit(X\_scaled)}
\NormalTok{    inertia.append(kmeans.inertia\_)}

\CommentTok{\# Plot the elbow method to decide on the best \textquotesingle{}K\textquotesingle{}}
\NormalTok{plt.figure(figsize}\OperatorTok{=}\NormalTok{(}\DecValTok{8}\NormalTok{, }\DecValTok{5}\NormalTok{))}
\NormalTok{plt.plot(}\BuiltInTok{range}\NormalTok{(}\DecValTok{1}\NormalTok{, }\DecValTok{11}\NormalTok{), inertia, marker}\OperatorTok{=}\StringTok{\textquotesingle{}o\textquotesingle{}}\NormalTok{)}
\NormalTok{plt.title(}\StringTok{\textquotesingle{}Determining the Number of Clusters (K)\textquotesingle{}}\NormalTok{)}
\NormalTok{plt.xlabel(}\StringTok{\textquotesingle{}Clusters\textquotesingle{}}\NormalTok{)}
\NormalTok{plt.ylabel(}\StringTok{\textquotesingle{}Inertia\textquotesingle{}}\NormalTok{)}
\NormalTok{plt.show()}
\end{Highlighting}
\end{Shaded}

\begin{figure}[H]

\sidecaption{Determining the Optimal K with Elbow Method}

{\centering \pandocbounded{\includegraphics[keepaspectratio]{clustering_files/figure-pdf/cell-7-output-1.pdf}}

}

\end{figure}%

\hfill\break
Now that we have determined the optimal k, we can proceed to create the
kmeans model using \(k=5\).\\

\begin{Shaded}
\begin{Highlighting}[]
\CommentTok{\# Apply K{-}means with K=5}
\NormalTok{kmeans }\OperatorTok{=}\NormalTok{ KMeans(n\_clusters}\OperatorTok{=}\DecValTok{5}\NormalTok{, random\_state}\OperatorTok{=}\DecValTok{626}\NormalTok{)}
\NormalTok{kmeans.fit(X\_scaled)}
\end{Highlighting}
\end{Shaded}

{
\makeatletter
\def\LT@makecaption#1#2#3{%
  \noalign{\smash{\hbox{\kern\textwidth\rlap{\kern\marginparsep
  \parbox[t]{\marginparwidth}{%
    \footnotesize{%
      \vspace{(1.1\baselineskip)}
    #1{#2: }\ignorespaces #3}}}}}}%
    }
\makeatother

\begin{longtable}[]{@{}
  >{\raggedright\arraybackslash}p{(\linewidth - 4\tabcolsep) * \real{0.3333}}
  >{\raggedright\arraybackslash}p{(\linewidth - 4\tabcolsep) * \real{0.3333}}
  >{\raggedright\arraybackslash}p{(\linewidth - 4\tabcolsep) * \real{0.3333}}@{}}
\toprule\noalign{}
\endhead
\bottomrule\noalign{}
\endlastfoot
\emph{} & \begin{minipage}[t]{\linewidth}\raggedright
\href{https://scikit-learn.org/1.8/modules/generated/sklearn.cluster.KMeans.html\#:~:text=n_clusters,-int\%2C\%20default\%3D8}{n\_clusters
{n\_clusters: int, default=8\\
\strut \\
The number of clusters to form as well as the number of\\
centroids to generate.\\
\strut \\
For an example of how to choose an optimal value for `n\_clusters` refer
to\\
:ref:`sphx\_glr\_auto\_examples\_cluster\_plot\_kmeans\_silhouette\_analysis.py`.}}\strut
\end{minipage} & 5 \\
\emph{} & \begin{minipage}[t]{\linewidth}\raggedright
\href{https://scikit-learn.org/1.8/modules/generated/sklearn.cluster.KMeans.html\#:~:text=init,-\%7B\%27k-means\%2B\%2B\%27\%2C\%20\%27random\%27\%7D\%2C\%20callable\%20or\%20array-like\%20of\%20shape\%20\%20\%20\%20\%20\%20\%20\%20\%20\%20\%20\%20\%20\%28n_clusters\%2C\%20n_features\%29\%2C\%20default\%3D\%27k-means\%2B\%2B\%27}{init
{init: \{\textquotesingle k-means++\textquotesingle,
\textquotesingle random\textquotesingle\}, callable or array-like of
shape (n\_clusters, n\_features),
default=\textquotesingle k-means++\textquotesingle{}\\
\strut \\
Method for initialization:\\
\strut \\
* \textquotesingle k-means++\textquotesingle{} : selects initial cluster
centroids using sampling based on an empirical probability distribution
of the points\textquotesingle{} contribution to the overall inertia.
This technique speeds up convergence. The algorithm implemented is
"greedy k-means++". It differs from the vanilla k-means++ by making
several trials at each sampling step and choosing the best centroid
among them.\\
\strut \\
* \textquotesingle random\textquotesingle: choose `n\_clusters`
observations (rows) at random from data for the initial centroids.\\
\strut \\
* If an array is passed, it should be of shape (n\_clusters,
n\_features) and gives the initial centers.\\
\strut \\
* If a callable is passed, it should take arguments X, n\_clusters and a
random state and return an initialization.\\
\strut \\
For an example of how to use the different `init` strategies, see\\
:ref:`sphx\_glr\_auto\_examples\_cluster\_plot\_kmeans\_digits.py`.\\
\strut \\
For an evaluation of the impact of initialization, see the example\\
:ref:`sphx\_glr\_auto\_examples\_cluster\_plot\_kmeans\_stability\_low\_dim\_dense.py`.}}\strut
\end{minipage} & \textquotesingle k-means++\textquotesingle{} \\
\emph{} & \begin{minipage}[t]{\linewidth}\raggedright
\href{https://scikit-learn.org/1.8/modules/generated/sklearn.cluster.KMeans.html\#:~:text=n_init,-\%27auto\%27\%20or\%20int\%2C\%20default\%3D\%27auto\%27}{n\_init
{n\_init: \textquotesingle auto\textquotesingle{} or int,
default=\textquotesingle auto\textquotesingle{}\\
\strut \\
Number of times the k-means algorithm is run with different centroid\\
seeds. The final results is the best output of `n\_init` consecutive
runs\\
in terms of inertia. Several runs are recommended for sparse\\
high-dimensional problems (see :ref:`kmeans\_sparse\_high\_dim`).\\
\strut \\
When `n\_init=\textquotesingle auto\textquotesingle`, the number of runs
depends on the value of init:\\
10 if using `init=\textquotesingle random\textquotesingle` or `init` is
a callable;\\
1 if using `init=\textquotesingle k-means++\textquotesingle` or `init`
is an array-like.\\
\strut \\
.. versionadded:: 1.2\\
Added \textquotesingle auto\textquotesingle{} option for `n\_init`.\\
\strut \\
.. versionchanged:: 1.4\\
Default value for `n\_init` changed to
`\textquotesingle auto\textquotesingle`.}}\strut
\end{minipage} & \textquotesingle auto\textquotesingle{} \\
\emph{} & \begin{minipage}[t]{\linewidth}\raggedright
\href{https://scikit-learn.org/1.8/modules/generated/sklearn.cluster.KMeans.html\#:~:text=max_iter,-int\%2C\%20default\%3D300}{max\_iter
{max\_iter: int, default=300\\
\strut \\
Maximum number of iterations of the k-means algorithm for a\\
single run.}}\strut
\end{minipage} & 300 \\
\emph{} & \begin{minipage}[t]{\linewidth}\raggedright
\href{https://scikit-learn.org/1.8/modules/generated/sklearn.cluster.KMeans.html\#:~:text=tol,-float\%2C\%20default\%3D1e-4}{tol
{tol: float, default=1e-4\\
\strut \\
Relative tolerance with regards to Frobenius norm of the difference\\
in the cluster centers of two consecutive iterations to declare\\
convergence.}}\strut
\end{minipage} & 0.0001 \\
\emph{} & \begin{minipage}[t]{\linewidth}\raggedright
\href{https://scikit-learn.org/1.8/modules/generated/sklearn.cluster.KMeans.html\#:~:text=verbose,-int\%2C\%20default\%3D0}{verbose
{verbose: int, default=0\\
\strut \\
Verbosity mode.}}\strut
\end{minipage} & 0 \\
\emph{} & \begin{minipage}[t]{\linewidth}\raggedright
\href{https://scikit-learn.org/1.8/modules/generated/sklearn.cluster.KMeans.html\#:~:text=random_state,-int\%2C\%20RandomState\%20instance\%20or\%20None\%2C\%20default\%3DNone}{random\_state
{random\_state: int, RandomState instance or None, default=None\\
\strut \\
Determines random number generation for centroid initialization. Use\\
an int to make the randomness deterministic.\\
See :term:`Glossary `.}}\strut
\end{minipage} & 626 \\
\emph{} & \begin{minipage}[t]{\linewidth}\raggedright
\href{https://scikit-learn.org/1.8/modules/generated/sklearn.cluster.KMeans.html\#:~:text=copy_x,-bool\%2C\%20default\%3DTrue}{copy\_x
{copy\_x: bool, default=True\\
\strut \\
When pre-computing distances it is more numerically accurate to center\\
the data first. If copy\_x is True (default), then the original data
is\\
not modified. If False, the original data is modified, and put back\\
before the function returns, but small numerical differences may be\\
introduced by subtracting and then adding the data mean. Note that if\\
the original data is not C-contiguous, a copy will be made even if\\
copy\_x is False. If the original data is sparse, but not in CSR
format,\\
a copy will be made even if copy\_x is False.}}\strut
\end{minipage} & True \\
\emph{} & \begin{minipage}[t]{\linewidth}\raggedright
\href{https://scikit-learn.org/1.8/modules/generated/sklearn.cluster.KMeans.html\#:~:text=algorithm,-\%7B\%22lloyd\%22\%2C\%20\%22elkan\%22\%7D\%2C\%20default\%3D\%22lloyd\%22}{algorithm
{algorithm: \{"lloyd", "elkan"\}, default="lloyd"\\
\strut \\
K-means algorithm to use. The classical EM-style algorithm is
`"lloyd"`.\\
The `"elkan"` variation can be more efficient on some datasets with\\
well-defined clusters, by using the triangle inequality. However
it\textquotesingle s\\
more memory intensive due to the allocation of an extra array of shape\\
`(n\_samples, n\_clusters)`.\\
\strut \\
.. versionchanged:: 0.18\\
Added Elkan algorithm\\
\strut \\
.. versionchanged:: 1.1\\
Renamed "full" to "lloyd", and deprecated "auto" and "full".\\
Changed "auto" to use "lloyd" instead of "elkan".}}\strut
\end{minipage} & \textquotesingle lloyd\textquotesingle{} \\
\end{longtable}

}

\begin{Shaded}
\begin{Highlighting}[]
\CommentTok{\# Add the cluster identifiers as a new field in the dataset}
\NormalTok{df }\OperatorTok{=}\NormalTok{ df.with\_columns(}
\NormalTok{    pl.Series(name}\OperatorTok{=}\StringTok{\textquotesingle{}Cluster\textquotesingle{}}\NormalTok{, values}\OperatorTok{=}\NormalTok{kmeans.labels\_)}
\NormalTok{)}
\end{Highlighting}
\end{Shaded}

\section{Model Visualization}\label{model-visualization}

\begin{Shaded}
\begin{Highlighting}[]
\NormalTok{plt.figure(figsize}\OperatorTok{=}\NormalTok{(}\DecValTok{10}\NormalTok{, }\DecValTok{6}\NormalTok{))}
\NormalTok{sns.scatterplot(x}\OperatorTok{=}\NormalTok{df[}\StringTok{\textquotesingle{}Annual Income (k$)\textquotesingle{}}\NormalTok{],}
\NormalTok{                y}\OperatorTok{=}\NormalTok{df[}\StringTok{\textquotesingle{}Spending Score (1{-}100)\textquotesingle{}}\NormalTok{],}
\NormalTok{                hue}\OperatorTok{=}\NormalTok{df[}\StringTok{\textquotesingle{}Cluster\textquotesingle{}}\NormalTok{],}
\NormalTok{                palette}\OperatorTok{=}\StringTok{\textquotesingle{}viridis\textquotesingle{}}\NormalTok{,}
\NormalTok{                s}\OperatorTok{=}\DecValTok{100}\NormalTok{)}
\NormalTok{plt.scatter(X[:, }\DecValTok{0}\NormalTok{], X[:, }\DecValTok{1}\NormalTok{], c}\OperatorTok{=}\NormalTok{df[}\StringTok{\textquotesingle{}Cluster\textquotesingle{}}\NormalTok{], cmap}\OperatorTok{=}\StringTok{\textquotesingle{}viridis\textquotesingle{}}\NormalTok{)}
\NormalTok{plt.title(}\StringTok{\textquotesingle{}Customer Segments based on income and spending Score\textquotesingle{}}\NormalTok{)}
\NormalTok{plt.xlabel(}\StringTok{\textquotesingle{}Annual Income\textquotesingle{}}\NormalTok{) }
\NormalTok{plt.ylabel(}\StringTok{\textquotesingle{}Spending Score\textquotesingle{}}\NormalTok{)}
\NormalTok{plt.legend()}
\NormalTok{plt.show()}
\end{Highlighting}
\end{Shaded}

\begin{figure}[H]

\sidecaption{Visualizing the clusters using annual income and spending
score}

{\centering \pandocbounded{\includegraphics[keepaspectratio]{clustering_files/figure-pdf/cell-10-output-1.pdf}}

}

\end{figure}%

\chapter{Conclusions}\label{conclusions}

This document demonstrated a basic application of K-Means clustering in
Python.

We hope this provides a clear understanding of the fundamental steps
involved in using this powerful unsupervised learning technique.

Experimentation is key in clustering as we can see when exploring with
different values of k, and visualize results to gain deeper insights
from data.

Clustering is a valuable tool for data exploration and discovery. By
uncovering hidden structures in data, clustering can lead to valuable
insights and inform decision-making in various domains.

\chapter{Reference}\label{reference}

\begin{itemize}
\tightlist
\item
  Carrascosa, Ivan Palomares. 2025.
  ``\href{https://machinelearningmastery.com/the-beginners-guide-to-clustering-with-python}{The
  Beginner's Guide to Clustering with Python}.'' Practical Machine
  Learning.
\end{itemize}

\chapter{Contact}\label{contact}

\textbf{Jesus L. Monroy}\\
\emph{Economist \& Data Scientist}

\href{https://www.linkedin.com/in/j3sus-lm}{Linkedin} \textbar{}
\href{https://medium.com/@jesuslm}{Medium} \textbar{}
\href{https://x.com/j3suslm}{Twitter}





\end{document}
